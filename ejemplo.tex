\documentclass[]{tesisusach}

\usepackage{lipsum}

\title{Neque porro quisquam est qui dolorem ipsum quia dolor sit amet}
\author{Fulano de Tal \and Perengano los Palotes}
\date{\today}

\keywordsEs{Prueba, Tesis}
\keywordsEn{Test, Thesis}

%\universidad{universidad de santiago de chile}
\subtitle{Consectetur, adipisci, velit\ldots}

\facultad{facultad de ingeniería}
\unidad{Departamento de Ingeniería Informática}
\profesor{Mengano}
\profesorco{Sutano}
\propuesta{Propuesta de Tesis para optar por el grado de Doctor en Ciencias de la Ingeniería con mención en Informática}
\ciudad{Santiago}
\pais{Chile}

\addbibresource{caps/referencias.bib}

\begin{document}
	\maketitle
	% ----------------------------------------------------------
	% ----------- PRIMERA PARTE --------------------------------
	% Temas preliminares: abstract, agradecimientos, 
	% dedicatorias...
	\frontmatter
	
	\begin{resumenEs}
		\lipsum[1]	
	\end{resumenEs}
	
	\begin{resumenEn}
		\lipsum[1]	
	\end{resumenEn}

	\dedicatoriaSimple{A Lilith, señora de los desposeídos}
	
	\begin{agradecimiento}
		\lipsum[2-3]
	\end{agradecimiento}

	% De acuerdo al formato, lo preliminar termina con las tablas
	% de contenido.	
	\tableofcontents
	\newpage
	%% Indice de tablas
	\listoftables
	\newpage
	%% Indice de figuras
	\listoffigures
	\newpage

	% ----------------------------------------------------------
	% ----------- SEGUNDA PARTE --------------------------------
	% Los capítulos solo están de relleno.
	\mainmatter
	% Archivo de ejemplo para colocar introducción
\chapter{Introducción}
\lipsum[4] Ver \cite{beile2005development}.

\lipsum*[5][1-2]\citep{ala1989presidential}, \lipsum[6] 

\lipsum[7-10]
	% Capítulo de ejemplo para colocar descripción del problema
\chapter{Problema}
\lipsum[11-15]
\begin{itemize}
\item \lipsum[1][1]
\item \lipsum[1][2]
\item \lipsum[1][3]
\end{itemize}

\lipsum[16-17]
\begin{enumerate}
\item \lipsum[2][1]
\item \lipsum[2][2]
\item \lipsum[2][3]
\end{enumerate}

\lipsum[18]
\begin{enumerate}[i)]
\item \lipsum[3][1]
\item \lipsum[3][2]
\item \lipsum[3][3]
\end{enumerate}

\lipsum[19-22]
\begin{enumerate}[(I)]
\item \lipsum[4][1]
\item \lipsum[4][2]
\item \lipsum[4][3]
\end{enumerate}
	\include{caps/03-conclusions}
	
	% ----------------------------------------------------------
	% ----------- TERCERA PARTE --------------------------------
	% ### Bibliografía de este documento ###
	% La opción heading la coloca en la tabla de contenidos
	\printbibliography[heading=bibintoc]
	
	\appendix
	%\addappheadtotoc
	\chapter{Addendum}

\lipsum[30]

\begin{table}[hbtp]
	\centering
	\begin{tabular}{|c|c|c|c|c|} \hline
        S & A & T & O & R\\ \hline
        A & R & E & P & O\\ \hline
        T & E & N & E & T\\ \hline
        O & P & E & R & A\\ \hline
        R & O & T & A & S\\ \hline
    \end{tabular}
    
    \caption{Dummy table}
    \label{tab:apptable}
\end{table}

\begin{figure}[hbtp]
	\centering
	\begin{tabular}{|c|c|c|c|c|} \hline
        S & A & T & O & R\\ \hline
        A & R & E & P & O\\ \hline
        T & E & N & E & T\\ \hline
        O & P & E & R & A\\ \hline
        R & O & T & A & S\\ \hline
    \end{tabular}
    
    \caption{Dummy fig}
    \label{fig:appfig}
\end{figure}

\lipsum[31-35]


\end{document}