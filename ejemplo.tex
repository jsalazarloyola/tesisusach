\documentclass{tesisusach}

\usepackage{lipsum}

\title{Test title}
\author{Fulano de Tal}
\date{\today}

\keywordsEs{Prueba, Tesis}
\keywordsEn{Test, Thesis}

\begin{document}
	\maketitle
	
	% ----------------------------------------------------------
	% ----------- PRIMERA PARTE --------------------------------
	% Temas preliminares: abstract, agradecimientos, 
	% dedicatorias...
	\frontmatter
	
	\begin{resumenEs}
		\lipsum[1]	
	\end{resumenEs}
	
	\begin{resumenEn}
		\lipsum[1]	
	\end{resumenEn}

	\dedicatoriaSimple{A Lilith, señora de los desposeídos}
	
	\begin{agradecimiento}
		\lipsum[2-3]
	\end{agradecimiento}

	% De acuerdo al formato, lo preliminar termina con las tablas
	% de contenido.	
	\tableofcontents
	\newpage
	%% Indice de tablas
	\listoftables
	\newpage
	%% Indice de figuras
	\listoffigures
	\newpage

	% ----------------------------------------------------------
	% ----------- SEGUNDA PARTE --------------------------------
	% Los capítulos solo están de relleno.
	\mainmatter
	
\end{document}