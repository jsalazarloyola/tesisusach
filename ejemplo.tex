\documentclass{tesisusach}

\usepackage{lipsum}

\title{Test title}
\author{Fulano de Tal}
\date{\today}

\keywordsEs{Prueba, Tesis}
\keywordsEn{Test, Thesis}

\begin{document}
	\maketitle
	
	% ----------------------------------------------------------
	% ----------- PRIMERA PARTE --------------------------------
	% Temas preliminares: abstract, agradecimientos, 
	% dedicatorias...
	\frontmatter
	
	\begin{resumenEs}
		\lipsum[1]	
	\end{resumenEs}
	
	\begin{resumenEn}
		\lipsum[1]	
	\end{resumenEn}

	\dedicatoriaSimple{A Lilith, señora de los desposeídos}
	
	\begin{agradecimiento}
		\lipsum[2-3]
	\end{agradecimiento}

	% De acuerdo al formato, lo preliminar termina con las tablas
	% de contenido.	
	\tableofcontents
	\newpage
	%% Indice de tablas
	\listoftables
	\newpage
	%% Indice de figuras
	\listoffigures
	\newpage

	% ----------------------------------------------------------
	% ----------- SEGUNDA PARTE --------------------------------
	% Los capítulos solo están de relleno.
	\mainmatter
	% Archivo de ejemplo para colocar introducción
\chapter{Introducción}
\lipsum[4-10]

	% Capítulo de ejemplo para colocar descripción del problema
\chapter{Problema}
\lipsum[11-15] \cite[Ver][pág.1250]{beile2007ilas}
\begin{itemize}
\item \lipsum[1][1]
\item \lipsum[1][2]
\item \lipsum[1][3]
\end{itemize}

\lipsum[16-17]
\begin{enumerate}
\item \lipsum[2][1] \parencite[Ver][]{boh2016development}
\item \lipsum[2][2]
\item \lipsum[2][3] Ver \ref{fig:dummyfigure1}.
\end{enumerate}

\begin{figure}[htbp]
	\centering
	\begin{tabular}{|c|c|c|c|c|} \hline
        S & A & T & O & R\\ \hline
        A & R & E & P & O\\ \hline
        T & E & N & E & T\\ \hline
        O & P & E & R & A\\ \hline
        R & O & T & A & S\\ \hline
    \end{tabular}
    
    \caption{Dummy figure}
    \label{fig:dummyfigure1}
\end{figure}

\lipsum[18]
\begin{enumerate}[i)]
\item \lipsum[3][1]
\item \lipsum[3][2] Ver \ref{fig:dummyfigure2}.
\item \lipsum[3][3] \parencite[][pág. 75]{bruce1997seven}
\end{enumerate}

\begin{figure}[htbp]
	\centering
	\begin{tabular}{|c|c|c|c|c|} \hline
        S & A & T & O & R\\ \hline
        A & R & E & P & O\\ \hline
        T & E & N & E & T\\ \hline
        O & P & E & R & A\\ \hline
        R & O & T & A & S\\ \hline
    \end{tabular}
    
    \caption{Dummy figure}
    \label{fig:dummyfigure2}
\end{figure}

\lipsum[19-22]
\begin{enumerate}[(I)]
\item \lipsum[4][1]
\item \lipsum[4][2]
\item \lipsum[4][3]
\end{enumerate}

\begin{table}[htbp]
	\centering
	\begin{tabular}{|c|c|c|c|c|} \hline
        S & A & T & O & R\\ \hline
        A & R & E & P & O\\ \hline
        T & E & N & E & T\\ \hline
        O & P & E & R & A\\ \hline
        R & O & T & A & S\\ \hline
    \end{tabular}
    
    \caption{Dummy table}
    \label{tab:dummytable1}
\end{table}

\lipsum[23-26]

\begin{table}[hbtp]
	\centering
	\begin{tabular}{|c|c|c|c|c|} \hline
        S & A & T & O & R\\ \hline
        A & R & E & P & O\\ \hline
        T & E & N & E & T\\ \hline
        O & P & E & R & A\\ \hline
        R & O & T & A & S\\ \hline
    \end{tabular}
    
    \caption[Dummy table de nombre largo]{Dummy table con un nombre ridículamente largo, porque a veces estas cosas ocurren y es necesario tratar estos casos, para poder dar un buen ejemplo}
    \label{tab:dummytable2}
\end{table}

\begin{table}[htbp]
	\centering
	\begin{tabular}{|c|c|c|c|c|} \hline
        S & A & T & O & R\\ \hline
        A & R & E & P & O\\ \hline
        T & E & N & E & T\\ \hline
        O & P & E & R & A\\ \hline
        R & O & T & A & S\\ \hline
    \end{tabular}
    
    \caption[Nombre corto]{Otra dummy table \parencite{cabrera2015evolucion}.}
    \label{tab:dummytable3}
\end{table}

\lipsum[27-30]

Probando diferentes comandos para referenciar: ver \rangeref{tab:dummytable1}{tab:dummytable3} o la  \Fig{fig:dummyfigure2}.
\end{document}